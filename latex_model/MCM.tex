\documentclass[12pt]{article}
\usepackage{lastpage} % 获取总页数
% 设置页面
\usepackage{geometry}
\geometry{left=1in,right=0.75in,top=1in,bottom=1in}
% 插图片
\usepackage{graphicx}
\usepackage{subcaption}
% 添加标题
\usepackage{caption}
% 设置字体
\usepackage{fontspec}
\setmainfont{Times New Roman}
\setmonofont{Courier New}
\usepackage{longtable}
% 题型,队伍号
\newcommand{\Problem}{C}
\newcommand{\Team}{1111111}
\usepackage{hyperref}
\hypersetup{
    colorlinks=true,  % 文字颜色(而非框线)
    linkcolor=blue,   % 内部引用(公式/表格/图片)颜色
    urlcolor=blue     % 网址链接颜色
}

\usepackage{amsmath,amssymb,amsthm,array,booktabs}
% 设置页眉页脚
\usepackage{fancyhdr}
\lhead{Team \Team}  % 页眉左侧:显示\Team
\rhead{}            % 页眉右侧:清空
\cfoot{}            % 页脚中间:清空
% 定义“定理”环境,标题显示“Theorem”
\newtheorem{theorem}{Theorem}
% 定义“推论”环境,编号和theorem共享
\newtheorem{corollary}[theorem]{Corollary}
% 定义“引理”环境,编号也和theorem共享
\newtheorem{lemma}[theorem]{Lemma}
% 定义“定义”环境,独立编号
\newtheorem{definition}{Definition}
% 颜色
\usepackage{xcolor}
% 代码高亮方案宏包
\usepackage{listings}
\definecolor{CPPLight}  {HTML} {686868}
\definecolor{CPPSteel}  {HTML} {888888}
\definecolor{CPPDark}   {HTML} {262626}
\definecolor{CPPBlue}   {HTML} {4172A3}
\definecolor{CPPGreen}  {HTML} {487818}
\definecolor{CPPBrown}  {HTML} {A07040}
\definecolor{CPPRed}    {HTML} {AD4D3A}
\definecolor{CPPViolet} {HTML} {7040A0}
\definecolor{CPPGray}  {HTML} {B8B8B8}
\lstset
{
	basicstyle=\ttfamily,        % 等宽代码字体
	breaklines=true,             % 代码超长自动换行
	framextopmargin=50pt,        % 代码块顶部留白
	frame=none,                  % 无边框
	columns=fixed,               % 等宽字符,排版整齐
    numbers=left,                % 在左侧显示行号
	backgroundcolor=\color[RGB]{255,255,255}, % 纯白背景
	showstringspaces=false,      % 不显示字符串中的空格圆点
	keywordstyle=\color{CPPBlue},          % 关键字:深蓝色
	commentstyle=\itshape\color{CPPGreen}, % 注释:斜体+深青色
	stringstyle=\slshape\color{CPPRed},    % 字符串:倾斜+深红色
	emphstyle=\color{CPPViolet},           % 强调词:紫色
	numberstyle=\footnotesize\color{darkgray}, % 行号格式
}
\lstdefinelanguage{Python}
{
    morekeywords={alignas,continute,friend,register,true,alignof,decltype,goto,
		reinterpret_cast,try,asm,defult,if,return,typedef,auto,delete,inline,short,
		typeid,bool,do,int,signed,typename,break,double,long,sizeof,union,case,
		dynamic_cast,mutable,static,unsigned,catch,else,namespace,static_assert,using,
		char,enum,new,static_cast,virtual,char16_t,char32_t,explict,noexcept,struct,
		void,export,nullptr,switch,volatile,class,extern,operator,template,wchar_t,
		const,false,private,this,while,constexpr,float,protected,thread_local,
		const_cast,for,public,throw,std},
	emph={map,set,multimap,multiset,unordered_map,unordered_set,numpy,graph,path,append,extend,
		unordered_multiset,unordered_multimap,vector,string,list,deque,
		array,stack,forwared_list,iostream,memory,shared_ptr,unique_ptr,
		random,bitset,ostream,istream,cout,cin,endl,move,default_random_engine,
		uniform_int_distribution,iterator,algorithm,functional,bing,numeric,},
	emphstyle=\color{CPPViolet},
}
\lstdefinelanguage{Matlab}
{
    morekeywords={if,elseif,else,end,for,while,switch,case,otherwise,function,return,
        classdef,properties,methods,global,persistent,try,catch,warning,error,true,false},
    emph={zeros,ones,rand,randn,linspace,meshgrid,size,length,reshape,plot,plot3,
        figure,xlabel,ylabel,zlabel,title,legend,grid,ode45,dsolve,subs,vpa,inv,eig,det,
        cell,struct,char,double,disp,fprintf,input,optimset,fmincon},
    emphstyle=\color{CPPViolet},
}

\begin{document}
\graphicspath{{.}}  % Place your graphic files in the same directory as your main document
\DeclareGraphicsExtensions{.pdf, .jpg, .tif, .png}
\thispagestyle{empty}
\vspace*{-16ex}
\centerline
{
\begin{tabular}{*3{c}}
	\parbox[t]{0.3\linewidth}{\begin{center}\textbf{Problem Chosen}\\ \Large \textcolor{red}{\Problem}\end{center}}
	& \parbox[t]{0.3\linewidth}{\begin{center}\textbf{2026\\ MCM/ICM\\ Summary Sheet}\end{center}}
	& \parbox[t]{0.3\linewidth}{\begin{center}\textbf{Team Control Number}\\ \Large \textcolor{red}{\Team}\end{center}}	\\
	\hline
\end{tabular}
}
\begin{center}
\textcolor{red}
{
Use this template to begin typing the first page (summary page) of your electronic report. This \newline
template uses a 12-point Times New Roman font. Submit your paper as an Adobe PDF \newline
electronic file (e.g. 1111111.pdf), typed in English, with a readable font of at least 12-point type.	\\[2ex]
Do not include the name of your school, advisor, or team members on this or any page.	\\[2ex]
Be sure to change the control number and problem choice above.	\\
You may delete these instructions as you begin to type your report here. 	\\[2ex]
\textbf{Follow us @COMAPMath on X or COMAPCHINAOFFICIAL on Weibo for the \newline
most up to date contest information.}
}
\end{center}
\clearpage %强制分页
\pagestyle{fancy} %启用自定义美化页眉页脚样式,关闭 LaTeX 默认的原始样式
\tableofcontents %生成目录
\newpage
\setcounter{page}{1} %正文页码从1开始
\rhead{Page \thepage\ of \pageref{LastPage}}  %页眉右侧显示当前页码/总页数
% 正文开始处

\begin{lstlisting}[language=Python, numbers=left, caption={Pythoncode}, label={code:python}]
import requests
import pandas as pd

# 真实API地址:世界银行教育数据(学生学习时长相关)
api_url = "http://api.worldbank.org/v2/country/all/indicator/SE.XPD.TOTL.GD.ZS?format=json&per_page=1000"

# 调用API(世界银行API需要加个参数指定语言)
response = requests.get(api_url, params={"lang": "en"})
data = response.json()[1]  # 提取数据部分

# 整理成表格,保存为CSV(美赛直接用)
df = pd.DataFrame(data)
df = df[["countryiso3code", "date", "value"]]  # 只保留核心列
df.columns = ["国家代码", "年份", "教育投入占GDP比例"]  # 重命名列(方便你看)
df.to_csv("C:/Users/17934/Desktop/美赛_世界银行教育数据.csv", index=False, encoding="utf-8")

print("数据获取成功!已保存为CSV文件")

\end{lstlisting}

\begin{lstlisting}[language=Matlab, numbers=left, caption={Matlabcode}, label={code:python}]
table = readtable('C:\Users\17934\Desktop\campu_data.csv');
x0=linspace(1,200,200);
y0=table.Study_Hours;
F1=griddedInterpolant(x0,y0,'linear'); %线性插值
F2=griddedInterpolant(x0,y0,'spline'); %三次样条插值
%F2=griddedInterpolant(x0,y0,'spline','extrap'); 对超出数据范围的插值运算使用外推方法
F3=griddedInterpolant(x0,y0,'cubic'); %三次多项式插值
x=1:0.1:200;
y1=F1(x);
y2=F2(x);
subplot(1,2,1);
plot(x,y1);
title('线性插值');
subplot(1,2,2);
plot(x,y2);
title('三次样条插值');

%三次样条插值的另一种计算方式-csape函数
%pp=csape(x0,y0);
%pp1=finder(pp); %计算pp函数的导数
%pp1=fnint(pp); %计算pp函数的积分
%y3=fnval(pp,x) %计算pp函数的函数值
\end{lstlisting}

\lstinputlisting[language=Python, numbers=left, caption={Python crawler code}, label={code:python_crawler}]{爬虫.py}

ready for MCM!
try\\
trytry\\
trytrytry\\

\section{Notations}

Here are all the notations and their meanings in this paper.
\begin{longtable}{>{\centering\arraybackslash}p{5em}>{\centering\arraybackslash}p{30em}}
\caption{Notations and Meanings}\label{tab:notation}\\
\toprule
Symbol & Meaning \\ \midrule
\endfirsthead

% 续表标题(可选,让第二页也有表头)
\caption{Notations and Meanings (Continued)}\\
\toprule
Symbol & Meaning \\ \midrule
\endhead

\bottomrule
\endfoot

$t$ & Time \\
$N$ & Total reported opioid cases\\
$N_t$ & Total reported drug cases\\
$\lambda$ & Average cases induced by a single case\\
$A_t$ & Status at $t$ \\ 
$E$ & Set containing socio-economic factors with high correlation $t$ \\
$T$ & Transition matrix\\ 
$i(t)$ & Proportion of opioid cases in all drug cases at $t$ \\
$\mu_1$ & Average number of drug cases induced by an existing drug case \\
$\mu_2$ & Number of opioid cases induced among all drug cases \\
$\gamma$ & Drug spread slow down factor \\
$i_0$ & Status at $t$ \\ 
$H$ & Information Entropy \\ 
$p_0$ & Initial number of drug cases\\
\end{longtable}

\begin{equation}
	N_t \frac{\mathrm{d}i}{\mathrm{d}t}=
	N_t \mu_2 i(t)(1-i(t)),i(0)=i_0 
	\label{SI}
\end{equation}
the equation~\ref{SI} above shows the dynamic equation of opioid spread.

\begin{center}
	\includegraphics[width=13cm,height=5.5cm]{piano.jpg}
    \captionof{figure}{Drug cases and proportion of opioid cases over time}\label{spread_rate}
\end{center}
the figure~\ref{spread_rate} above shows the drug cases and proportion of opioid cases over time.

try to put two figures side by side
\begin{figure}[!htbp]
    \centering
    % 第一个图片:占一行宽度的45%
    \begin{minipage}{0.45\textwidth}
        \centering
        \includegraphics[width=\textwidth]{piano.jpg} % 图片1路径
    \end{minipage}
    \hspace{0.05\textwidth} % 两张图之间的间距
    % 第二个图片:占一行宽度的45%
    \begin{minipage}{0.45\textwidth}
        \centering
        \includegraphics[width=\textwidth]{piano.jpg} % 图片2路径
    \end{minipage}
    % 共享标题
    \caption{Two pictures in one line (comparison)}
    \label{fig:two_pics}
\end{figure}

use different captions for each figure
\begin{figure}[!htbp]
    \centering
    % 第一张图+独立标题
    \begin{minipage}{0.45\textwidth}
        \centering
        \includegraphics[width=\textwidth]{piano.jpg}
        \caption{Picture 1}
        \label{fig:pic1}
    \end{minipage}
    \hspace{0.05\textwidth}
    % 第二张图+独立标题
    \begin{minipage}{0.45\textwidth}
        \centering
        \includegraphics[width=\textwidth]{piano.jpg}
        \caption{Picture 2}
        \label{fig:pic2}
    \end{minipage}
\end{figure}

fix location
\begin{center}
    \begin{minipage}{0.45\textwidth}
        \centering
        \includegraphics[width=\textwidth]{piano.jpg}
    \end{minipage}
    \hspace{0.05\textwidth}
    \begin{minipage}{0.45\textwidth}
        \centering
        \includegraphics[width=\textwidth]{piano.jpg}
    \end{minipage}
    \captionof{figure}{Two pictures in one line}\label{fig:two_pics}
\end{center}

\end{document}
